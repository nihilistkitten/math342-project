% !TEX TS-program = lualatex
\documentclass[11 pt]{amsart}

\usepackage[margin=1.3in]{geometry}

\usepackage{amssymb}
\usepackage{amsmath}
\usepackage{amsthm}
\usepackage{amsfonts}
\usepackage{amsxtra}
\usepackage{bbm}
\usepackage[all,2cell]{xy}
\usepackage{verbatim}
\usepackage{color}
\usepackage{float}
\usepackage{tikz-cd}
\usepackage{multicol}
\usepackage{quiver}
\usepackage{csquotes}
\usepackage[unicode=true, pdfusetitle,
 bookmarks=true,bookmarksnumbered=false,
 breaklinks=false,
 backref=false,
 colorlinks=true,
 linkcolor=blue,
 citecolor=blue,
 urlcolor=blue,
 final
]{hyperref}


\usepackage{bookmark}


\usepackage[capitalise]{cleveref}
\newcommand{\fref}{\cref}
\newcommand{\Fref}{\Cref}
\newcommand{\prettyref}{\cref}
\newcommand{\newrefformat}[2]{}



\theoremstyle{plain}   % This is the default, anyway
\newtheorem{thm}{Theorem}[section] % numbered theorem
\makeatletter\let\c@thm\c@thm\makeatother
\newtheorem{cor}{Corollary}[section]
\makeatletter\let\c@cor\c@thm\makeatother
\newtheorem{lemma}{Lemma}[section]
\makeatletter\let\c@lemma\c@thm\makeatother
\newtheorem{prop}{Proposition}[section]
\makeatletter\let\c@prop\c@thm\makeatother
\newtheorem{claim}{Claim}[section]
\makeatletter\let\c@claim\c@thm\makeatother



\newtheorem*{unnumberedtheorem}{Theorem}  % unnumbered theorem
\newtheorem*{unnumberedcorollary}{Corollary}
\newtheorem*{unnumberedlemma}{Lemma}
\newtheorem*{unnumberedproposition}{Proposition}
\newtheorem*{unnumberedclaim}{Claim}

\theoremstyle{definition}

\newtheorem{defn}{Definition}[section]
\makeatletter\let\c@defn\c@thm\makeatother
\newtheorem{const}{Construction}[section]
\makeatletter\let\c@const\c@thm\makeatother
\newtheorem{notn}{Notation}[section]
\makeatletter\let\c@notn\c@thm\makeatother
\newtheorem{outline}{Proof Outline}[section]
\makeatletter\let\c@outline\c@thm\makeatother


\theoremstyle{remark}

\newtheorem{rem}{Remark}[section]
\makeatletter\let\c@rem\c@thm\makeatother
\newtheorem{ex}{Example}[section]
\makeatletter\let\c@ex\c@thm\makeatother
\newtheorem{observation}{Observation}[section]
\makeatletter\let\c@observationn\c@thm\makeatother

\newtheorem*{unnumberedtheoremremark}{Remark}
\newtheorem*{unnumberedtheoremexample}{Example}
\newtheorem*{unnumberedtheoremdefinition}{Defintion}

\makeatletter
\let\c@equation\c@thm
\numberwithin{equation}{section}
\makeatother


 %Cleveref definitions
\crefname{lemma}{Lemma}{Lemmas}
\crefname{thm}{Theorem}{Theorems}
\crefname{defn}{Definition}{Definitions}
\crefname{notn}{Notation}{Notations}
\crefname{const}{Construction}{Constructions}
\crefname{prop}{Proposition}{Propositions}
\crefname{rem}{Remark}{Remarks}
\crefname{cor}{Corollary}{Corollaries}
\crefname{equation}{Diagram}{Diagrams}
\crefname{ex}{Example}{Examples}



%% ALPHABETS %%
\def\cA{\mathcal{A}}
\def\cB{\mathcal{B}}
\def\cC{\mathcal{C}}
\def\cD{\mathcal{D}}
\def\cE{\mathcal{E}}
\def\cF{\mathcal{F}}
\def\cG{\mathcal{G}}
\def\cH{\mathcal{H}}
\def\cI{\mathcal{I}}
\def\cJ{\mathcal{J}}
\def\cK{\mathcal{K}}
\def\cL{\mathcal{L}}
\def\cM{\mathcal{M}}
\def\cN{\mathcal{N}}
\def\cO{\mathcal{O}}
\def\cP{\mathcal{P}}
\def\cQ{\mathcal{Q}}
\def\cR{\mathcal{R}}
\def\cS{\mathcal{S}}
\def\cT{\mathcal{T}}
\def\cU{\mathcal{U}}
\def\cV{\mathcal{V}}
\def\cW{\mathcal{W}}
\def\cX{\mathcal{X}}
\def\cY{\mathcal{Y}}
\def\cZ{\mathcal{Z}}

\def\AA{\mathbb{A}}
\def\BB{\mathbb{B}}
\def\CC{\mathbb{C}}
\def\DD{\mathbb{D}}
\def\EE{\mathbb{E}}
\def\FF{\mathbb{F}}
\def\GG{\mathbb{G}}
\def\HH{\mathbb{H}}
\def\II{\mathbb{I}}
\def\JJ{\mathbb{J}}
\def\KK{\mathbb{K}}
\def\LL{\mathbb{L}}
\def\MM{\mathbb{M}}
\def\NN{\mathbb{N}}
\def\OO{\mathbb{O}}
\def\PP{\mathbb{P}}
\def\QQ{\mathbb{Q}}
\def\RR{\mathbb{R}}
\def\SS{\mathbb{S}}
\def\TT{\mathbb{T}}
\def\UU{\mathbb{U}}
\def\VV{\mathbb{V}}
\def\WW{\mathbb{W}}
\def\XX{\mathbb{X}}
\def\YY{\mathbb{Y}}
\def\ZZ{\mathbb{Z}}

%% OTHER SYMBOLS %% 
\DeclareMathOperator{\id}{id}
\DeclareMathOperator{\dom}{dom}
\DeclareMathOperator{\cod}{cod}

\def\nat{\Rightarrow}
\def\monic{\rightarrowtail}
\def\epic{\twoheadrightarrow}
\def\pathto{\rightsquigarrow}

\newcommand{\cat}[1]{{\normalfont\texttt{#1}}}
\newcommand{\op}[1]{{{#1}^{\cat{op}}}}
\newcommand{\opc}[1]{\op{\cat{#1}}}
\newcommand{\Obj}[1]{\text{Obj}(\cat{#1})}
\newcommand{\Map}[1]{\text{Map}(\cat{#1})}

% https://tex.stackexchange.com/a/118217
% \DeclarePairedDelimiter\ceil{\lceil}{\rceil}
% \DeclarePairedDelimiter\floor{\lfloor}{\rfloor}

\title{ Higher Category Theory by Way of Topology}

\author{Riley Shahar}

\pagestyle{plain}

\parindent0pt
\parskip4pt


\begin{document}



\begin{abstract}
	TODO
\end{abstract}


\maketitle

\section{Introduction}

TODO

After introducing some introductory category theory in \Cref{categories}, we
construct the fundamental groupoid in \Cref{the fundamental groupoid}. This
motivates natural transformations in \Cref{natural transformations}, in turn
leading to higher category theory in \Cref{higher categories}.

\section{Categorical Preliminaries}\label{categories}

Many of the constructions we'll encounter are categorical, or at least are
well-stated using categorical ideas, so it's worth spending some time with basic
category theory. Our aim will be to develop only some essential language now,
and later use our exploration of homotopy as motivation for developing higher
category theory. Except where noted, I follow \cite[Sections 1.1-1.3]{Riehl}.

\subsection{Categories}

\begin{defn}
	A \emph{category} $\cat{C}$ consists of a collection of \emph{objects} $\Obj{C}$
	and a collection of \emph{morphisms} $\Map{C}$, such that

	\begin{itemize}
		\item Each morphism $f$ has a specific \emph{domain} $\dom(f)$ and
		      \emph{codomain} $\cod(f)$; we write $f: x\rightarrow y$ or
		      $x\xrightarrow{f} y$.
		\item Each object $x$ has a specific \emph{identity morphism} $1_x:
			      x\rightarrow x$.
		\item For each pair of morphisms $x\xrightarrow{f} y\xrightarrow{g}
			      z$, there is a specific \emph{composite morphism} $gf:
			      x\rightarrow z$. We call $f$ and $g$ \emph{composition-compatible}.
	\end{itemize}
	This data must be \emph{associative} and \emph{unital}, which is to say,
	\begin{itemize}
		\item (Associativity) For composition-compatible triplet, i.e.
		      $x\xrightarrow{f}y \xrightarrow{g}z\xrightarrow{h}w$, we have $h(gf) =
			      (hg)f$. We write merely $hgf$.
		\item (Unital) For any $f: x\rightarrow y$, we have $1_yf = f = f1_x$.
	\end{itemize}

	In other words, the following diagrams commute:

	\begin{figure}[H]
		\centering
		\begin{multicols}{2}
			\begin{tikzcd}
				x & y & z & w
				\arrow["f", from=1-1, to=1-2]
				\arrow["g", from=1-2, to=1-3]
				\arrow["h", from=1-3, to=1-4]
				\arrow["gf"', curve={height=18pt}, from=1-1, to=1-3]
				\arrow["hg", curve={height=-18pt}, from=1-2, to=1-4]
			\end{tikzcd}

			\begin{tikzcd}
				x & x \\
				& y
				\arrow["1_x", from=1-1, to=1-2]
				\arrow["f", from=1-2, to=2-2]
				\arrow["f"', from=1-1, to=2-2]
			\end{tikzcd}
			\begin{tikzcd}
				x & y \\
				& y
				\arrow["f", from=1-1, to=1-2]
				\arrow["1_y", from=1-2, to=2-2]
				\arrow["f"', from=1-1, to=2-2]
			\end{tikzcd}
		\end{multicols}
	\end{figure}
\end{defn}


\begin{notn} We will use $\cat{C}(x, y)$ for the morphisms with domain $X$ and
	codomain $Y$. We also choose to use multiplicative notation for composition,
	instead of $\circ$, to distinguish it from function composition, since in many
	of our categories morphism composition will be concatenation or something
	completely separate from function composition. \end{notn}

\begin{ex}\label{concrete categories} Many mathematical structures can be studied as categories:
	\begin{itemize}
		\item $\cat{Set}$ is the category of sets and set-functions.
		\item $\cat{Poset}$ is the category of partially ordered sets and
		      monotone functions.
		\item $\cat{Grp}$ is the category of groups and homomorphisms.
		\item $\cat{Top}$ is the category of topological spaces and continuous
		      functions.
		\item For any field $\Bbbk$, $\cat{Vect}_\Bbbk$ is the category of vector
		      spaces over $k$ and linear maps.
	\end{itemize}
\end{ex}

These examples demonstrate the need for the word "collection" in the
definition of a category: there is no set of all sets, for example. In general,
we'll avoid detailing the set-theoretic issues here.

\begin{defn} A category is \emph{small} if its morphisms
	form a set. A category is \emph{locally small} if, for each pair of objects $x,
		y$, the morphisms $\cat{C}(x, y)$ forms a set. \end{defn}

\begin{rem}
	Any small category is locally small, since the morphisms between two objects
	form a subset of the set of morphisms. The objects in any small category form
	a set, since they are in bijection with the identity morphisms, which form a
	subset of the set of morphisms.
\end{rem}

None of the categories in \Cref{concrete categories} is small. In fact, each of
these examples takes as objects sets with certain structure and as morphisms
structure-preserving functions between those sets. As the following example
shows, these are far from the only kind of category.

\begin{ex}\label{abstract categories} The following are also categories:
	\begin{itemize}
		\item The \emph{empty category} is the category with no objects and no
		      morphisms.
		\item The \emph{trivial category} is the category with one object and only
		      its identity morphism.
		\item A set $X$ can be represented as a category whose objects correspond with the elements
		      of $X$ and which has only identity morphisms. A category with only
		      identity morphisms is called \emph{discrete}.
		\item A group $G$ can be represented as a category with one object whose
		      morphisms correspond with the element of $G$, with composition
		      determined by multiplication. Following \cite{Porter}, we call this
		      category $\GG$.
		\item A poset $P$ can be represented as a category whose objects correspond
		      with the elements of $P$ and with a single morphism $x\rightarrow y$
		      whenever $x\leq y$. We call this category $\PP$.
	\end{itemize}
\end{ex}

\subsection{(Iso)morphisms}

The main philosophy of category theory is roughly that

\begin{displayquote}
	an object is completely determined by its relationship with other objects,
	\cite[p. 8]{Bradley}
\end{displayquote}

and so to study an object, one should study its morphisms. I can't resist
sharing the following example:

\begin{ex}\cite[Example 2.1.6(ii)]{Riehl}
	Fixing a topological space $X$, we can recover both the points of $X$ and the
	topology on $X$---in other words, all the data of the space---via studying
	only maps involving $X$, as follows.

	To recover the points, note that a map $f: *\rightarrow X$, where $*$ is the
	singleton space, consists of picking out any point $f(*)\in X$. Thus the
	points are in bijection with $\cat{Top}(*, X)$.

	To recover the topology, let $S = \{0, 1\}$ with the topology $\{\emptyset,
		\{1\}, X\}$. To any open set $U\subseteq X$, associate the map
	$$f_U(x) = \begin{cases}
			1 & \text{ if }x\in U    \\
			0 & \text{ if }x\notin U
		\end{cases}$$

	But we can write any continuous function $f: X\rightarrow S$ in this form by
	setting $U = f^{-1}(\{1\})$. Thus the open sets are in bijection with
	$\cat{Top}(X, S)$.
\end{ex}

A chiefly important kind of morphism is the \emph{isomorphism}, which (as in
most algebric contexts) suggests a fundamental similarity between its domain and
codomain.

\begin{defn}
	A morphism $f: x\rightarrow y$ is an \emph{isomorphism} if there is a morphism
	$g: y\rightarrow x$ such that $gf = 1_x$ and $fg = 1_y$. We call $g$ the
	\emph{inverse} of $f$ and write $g = f^{-1}$. We say $x$ and $y$ are
	\emph{isomorphic} and write $x\cong y$.
\end{defn}

\begin{ex}\label{isomorphisms} Isomorphisms are interesting in many of our examples of categories:
	\begin{itemize}
		\item In $\cat{Set}$, the isomorphisms are invertible set-functions, i.e. bijections.
		\item In $\cat{Grp}$, the isomorphisms are group isomorphisms.
		\item In $\cat{Top}$, the isomorphisms are homeomorphisms.
		\item In $\cat{Vect}_{\Bbbk}$, the isomorphisms are vector space isomorphisms.
		\item For any group $G$, any morphism in $\GG$ is an isomorphism;
		      this fact corresponds to the existence of inverses in the group. (We
		      will have more to say about this example in \Cref{sec:groupoids}.)
		\item For any poset $P$, the isomorphisms in $\PP$ are the identities;
		      this fact corresponds to anti-symmetry of the relation.
	\end{itemize}
\end{ex}

In a locally small category $\cat{C}$, where $f: x\rightarrow y$ is any morphism
and $z$ is an ambient object, we can define maps $f_*: \cat{C}(z, x)\rightarrow
	\cat{C}(z, y)$ and $f^*: \cat{C}(y, z)\rightarrow \cat{C}(x, z)$ via post- and
pre-composition by $f$, respectively. In this case, \Cref{isomorphism
	characterization} gives an important characterization of isomorphisms in terms
of these maps.

\begin{thm}\label{isomorphism characterization}Let $\cat{C}$ be locally small.
	Then the following are equivalent:

	\begin{enumerate}
		\item  $f: x\rightarrow y$ is an isomorphism.
		\item  For every $z\in\cat{C}$, $f_*$ is a bijection of sets.
		\item For every $z\in\cat{C}$, $f^*$ is a bijection of sets.
	\end{enumerate}
\end{thm}

\begin{rem}
	If objects are characterized by their morphisms, then \Cref{isomorphism
		characterization} supports the idea that isomorphic objects truly look the
	same to the machinery of the category.
\end{rem}

\begin{proof}[Proof of \Cref{isomorphism characterization}]
	We prove equivalence $(1)\Leftrightarrow(2)$; the proof $(1)\Leftrightarrow(3)$ is
	similar\footnote{One can also conclude the latter via studying $(2)$ in the
		context of the \emph{opposite category} of $\cat{C}$, but this is outside our
		scope. See \cite[Lemma 1.2.3]{Riehl} for such a proof.}.

	Let $f$ be an isomorphism with inverse $g$. We show $g_*$ is an inverse of
	$f_*$. In particular, for any morphisms $h: z\rightarrow x$ and $k:
		z\rightarrow y$, we have that
	$$g_*(f_*(h)) = gfh = 1_Xh = h
		\quad
		\text{ and }
		\quad
		f_*(g_*(k)) = fgk = 1_Yk = k.$$

	Conversely, let $f_*$ be bijective. Letting $z = y$, by surjectivity there is
	some $g\in\cat{C}(y, x)$ such that $1_y = f_*(g) = fg$. But now letting $z =
		x$, we see that $$f_*(gf) = fgf = 1_yf = f = f_*(1_x),$$ and so by injectivity
	$gf = 1_x$. Thus $g$ is an inverse of $f$, hence $f$ is an isomorphism.
	\qedhere

\end{proof}

\subsection{Functors}

If the philosophy of category theory is to study morphisms between objects, then
functors answer the obvious question: what are the morphisms between categories?

\begin{defn}
	A \emph{functor} $F: \cat{C}\rightarrow\cat{D}$ between
	categories $\cat{C}$ and $\cat{D}$ consists of
	\begin{itemize}
		\item For each object $x\in\Obj{C}$, an object $Fx\in\Obj{D}$.
		\item For each morphism $f\in\Map{C}$, a morphism $Ff\in\Map{D}$.
	\end{itemize}

	This data must preserve the categorical structure, i.e. domains, codomains,
	identities, and composites, that is,

	\begin{itemize}
		\item For each $f\in\Map{C}$, $\dom(Ff) = F(\dom(f))$ and $\cod(Ff) =
			      F(\cod(f))$.
		\item For each composition-compatible pair $f,g\in\Map{C}$, $Fg\cdot Ff =
			      F(g\cdot f)$.
		\item For each $x\in\Obj{C}$, $F(1_x) = 1_{Fx}$.
	\end{itemize}
\end{defn}

\begin{ex}
	Many common constructions are functors:
	\begin{itemize}
		\item The power set defines a functor $\cP:
			      \cat{Set}\rightarrow\cat{Set}$ which takes a morphism $f$ to its
		      action via images, i.e. $\cP(f)(A) = f(A)$.
		\item On a category $\cat{C}$ like those in \Cref{concrete categories},
		      say $\cat{Grp}$, the \emph{forgetful functor}
		      $\cat{C}\rightarrow\cat{Set}$ sends an object to its underlying set
		      and a morphism to its underlying set-function,
		      "forgetting" the algebraic structure.
		\item The free group defines a functor $\cat{Set}\rightarrow\cat{Grp}$ which
		      sends a set to its free group and a map to its letter-wise action on
		      words.
		\item A group action of a group $G$ on a set $A$ can be regarded as a functor
		      $\GG\rightarrow\cat{Set}$ wich sends the single object of $\GG$ to $A$ and
		      a morphism to the endofunction defined by that element's action on
		      $A$.
	\end{itemize}
\end{ex}

\begin{thm}\label{functors preserve isomorphism}
	Let $F: \cat{C}\rightarrow \cat{D}$ be a functor and $f: x\rightarrow y$ an isomorphism in
	$\cat{D}$. Then $Ff$ is an isomorphism between $Fx$ and $Fy$ in $\cat{D}$.
\end{thm}

\begin{proof}
	Let $g: y\rightarrow x$ invert $f$. Then $$(Fg)(Ff) = F(gf) = F1_x = 1_{Fx},$$
	and the same works on the other side.
\end{proof}

There is a category of categories, $\cat{Cat}$, whose objects are categories and
whose morphisms are functors\footnote{Categories in \cat{Cat} must be locally
	small, for set-theoretic reasons. It is common, e.g. in \cite{Riehl}, to call
	this category $\cat{CAT}$, to distinguish it from the category of small
	categories, which in turn is called $\cat{Cat}$. As we ignore these details, we
	use the more convenient notation.}. This allows us to define isomorphic
categories straightforwardly:

\begin{defn}
	Two categories $\cat{C}$ and $\cat{D}$ are \emph{isomorphic} if there is a
	functor $F: \cat{C}\rightarrow\cat{D}$ which is an isomorphism in $\cat{Cat}$.
\end{defn}

We will see later, in \Cref{fundamental groupoid is not homotopy invariant}, that
this definition is generally too restrictive. That will motivate our discussion of
\emph{natural transformations} in \Cref{natural transformations}, where we will
define a more useful notion of \emph{equivalence of categories}.

\subsection{Groupoids}\label{sec:groupoids}

According to \Cref{abstract categories} and \Cref{isomorphisms}, any group can
be represented by a specific kind of category, one with a single object and only
isomorphisms. Indeed, any such category assembles into a group:

\begin{itemize}
	\item Elements are given by morphisms, with multiplication given by
	      composition.
	\item Multiplication is well-defined because any two morphisms have the same
	      domain and codomain (the single object), and hence are
	      composition-compatible.
	\item Multiplication is associative because composition is associative.
	\item The group identity is the identity morphism on the single object.
	\item Morphisms have inverses since they are isomorphisms.
\end{itemize}

It turns out that the requirement of a single object isn't necessary for the
algebraic structure we recover to be interesting. Indeed, relaxing this
requirement gives our fundamental object of study:

\begin{defn}
	A \emph{groupoid} is a category in which every morphism is an isomorphism.
\end{defn}

The second bullet above suggests that a groupoid will have to sacrifice
well-definideness of multiplication. Indeed, there is a purely algebraic picture
of groupoids defined in terms of partial functions, but the categorical one will
be sufficient for our needs. For a more complete discussion, see \cite{Brown}.

\begin{ex}\label{groupoids}
	The following are examples of groupoids:
	\begin{itemize}
		\item Per the above discussion, a group is a groupoid with one element.
		\item The categorification of a set, from \Cref{concrete categories}, is a
		      groupoid.
		\item The empty and trivial categories, from \Cref{abstract categories}, are
		      groupoids.
		\item A groupoid with only identity morphisms is a \emph{discrete groupoid}.
		\item A groupoid with at least one morphism between any two objects is \emph{connected}.
		\item A groupoid with precisely one morphism between any two objects is a
		      \emph{tree groupoid}.
	\end{itemize}
\end{ex}

Since groupoids are defined as categories, unsurprisingly, the morphisms between
them are functors:

\begin{defn}\cite[Section 6.4]{Brown}
	A \emph{groupoid morphism} is a functor between groupoids.
\end{defn}

Unsurprisingly, there is a category, $\cat{Grpd}$, of groupoids and groupoid
morphisms.

\section{The Fundamental Groupoid} \label{the fundamental groupoid}

A central idea of algebraic topology is that topological notions are encoded in
algebraic invariants of a space. The first such invariant we study is the
\emph{fundamental groupoid}, in some sense a more natural object than the
fundamental group, constructed from the homotopy classes of paths in a space.
Except where noted, I follow \cite[Chapter 6]{Bradley}.

\subsection{Homotopy}\label{sec:homotopy}

Recall that a \emph{homotopy} between continuous functions $f: X\rightarrow Y$
and $g: X\rightarrow Y$ is a continuous function $H: X\times I\rightarrow Y$
such that $H(-, 0) = f$ and $H(-, 1) = g$. This homotopy is further a \emph{path
	homotopy}\footnote{ When the context is unclear, we will call a general homotopy
	a \emph{free homotopy}, and a homotopy with fixed endpoints a \emph{path
		homotopy}. } when $X = I$ (so $f$ and $g$ are paths) and when $H(0, -)$ and
$H(1, -)$ are constant.

Two functions are \emph{free homotopic} when there exists a homotopy between
them. This forms an equivalence relation on the class of continuous functions
$X\rightarrow Y$; we write $f\simeq g$. When the functions are paths and the
homotopy is a path homotopy, the paths are \emph{path homotopic} and we write
$f\simeq_p g$.

Two spaces $X$ and $Y$ are \emph{homotopy equivalent} when there are maps $f:
	X\rightarrow Y$ and $g: Y\rightarrow X$ such that $fg\simeq 1_Y$ and $gf\simeq
	1_X$.

We can form the category $\cat{hTop}$ by taking spaces as objects and homotopy
classes of maps as morphisms. One needs to check that composition "works" in
this category; this was done in class.

\begin{rem} Two spaces are homotopy equivalent precisely they are isomorphic in
	$\cat{hTop}$. There is a functor $\cat{Top}\rightarrow\cat{hTop}$ that is the
	identity on objects and sends a continuous function to its path homotopy
	class. We therefore call functors out of $\cat{hTop}$ \emph{homotopy
		invariants}, since by \Cref{functors preserve isomorphism} they homotopy
	equivalent spaces to isomorphic objects. \end{rem}

\subsection{Construction}

Before defining the fundamental groupoid, there is a nice geometric picture to
tell. Fix a topological space $X$ and pick some points and paths between those
points. When you "erase" the other information of the underlying
space, you get several points and a bunch of (double-headed) arrows.

\begin{figure}[H]
	\centering
	\usetikzlibrary{decorations.markings}

\newcommand{\pathA}{(2, 2) .. controls (3,2) and (3, 1.5) .. (4, 1.5)}
\newcommand{\pathB}{(4, 1.5) .. controls (5, 1.5) and (5,2) .. (6, 2)}
\newcommand{\pathC}{(6, -2) .. controls (5,-2) and (5, -1.5) .. (4, -1.5)}
\newcommand{\pathD}{(4, -1.5) .. controls (3, -1.5) and (3,-2) .. (2, -2)}

\begin{tikzpicture}
	% outer shape
	\filldraw[fill=black, fill opacity=0.2, draw=black] (0, 0) arc(180:90:2) --
	\pathA --
	\pathB --
	(6, 2) arc(90:0:2) --
	(8, 0) arc(360:270:2) --
	\pathC --
	\pathD --
	(2, -2) arc(270:180:2)
	;

	% inner holes
	\filldraw [fill=white] (2,0) circle (1);
	\filldraw [fill=white] (6,0) circle (1);

	% paths
	\coordinate (x) at (3.4, 0.6);
	\node [circle, fill, inner sep=2pt] at (x) {};
	\draw[<->] ([shift=(37:1.5)]2,0) arc(37:370:1.5);
	\draw[<-] (3.65, 0.8) .. controls (4,1) and (5,1.5) .. (6, 1.5);
	\draw[<-] (3.7, 0.4) .. controls (4.5,0) and (5,-1.5) .. (6, -1.5);
	\draw ([shift=(270:1.5)]6,0) arc(270:450:1.5);

	\coordinate (x) at (12.4, 0.6);
	\node [circle, fill, inner sep=2pt] at (x) {};
	\draw[<->] ([shift=(37:1.5)]11,0) arc(37:370:1.5);
	\draw[<-] (12.65, 0.8) .. controls (13,1) and (14,1.5) .. (15, 1.5);
	\draw[<-] (12.7, 0.4) .. controls (13.5,0) and (14,-1.5) .. (15, -1.5);
	\draw ([shift=(270:1.5)]15,0) arc(270:450:1.5);
\end{tikzpicture}
% \begin{tikzpicture}[>={[inset=0,angle'=27]Stealth}]
% 	\draw circle(2);
% 	\draw [thick,fill=cyan!20](230:2)--(0,0)--(130:2) arc (130:230:2)--cycle;
% 	\draw [->](0,0)--node[above]{$r$} (10:2);
% 	\draw [|<->|](130:2.3) arc (130:230:2.3) node[left,pos=.5]{$L$};
% 	\draw [<->]  (130:1)   arc (130:230:1)   node[left,pos=.5]{$\theta$};
% \end{tikzpicture}
% \begin{tikzpicture}
% 	\draw (0,0) .. controls (1,1) and (2,-1) .. (0,0);
% 	% \begin{scope}[transparency group, opacity=0.6]
% 	% 	\path [draw=none, fill=gray] (-1.5,0) circle (2);
% 	% 	\path [draw=none, fill=gray] (1.5,0) circle (2);
% 	% \end{scope}
% \end{tikzpicture}

	\caption{Some points and paths on a two-holed disk.}
	\label{fig:group}
\end{figure}

This looks suspiciously like a groupoid---in particular, each of the arrows are
double-headed, and hence "invertible". The issue is that we don't necessarily
see the inverses or composites when we just take some points and some paths.

We'll return to this intuition of the fundamental groupoid "forgetting" some of
the underlying geometry of the space. More immediately, however, this view
strongly suggests that when we take all the path classes from our underlying space,
we should get back a groupoid.

\begin{defn}[Fundamental Groupoid]
	The \textit{fundamental groupoid} $\Pi_1X$ of a space $X$ is the category
	whose objects are points of $X$ and whose morphisms are path homotopy classes
	of paths in $X$.
\end{defn}

Specifically, let $x,y,z\in X$, $f$ a path from $x$ to $y$, and $g$ a path from
$y$ to $z$. Then,

\begin{itemize}
	\item A path's domain is its startpoint: $\dom([f]) = x$.
	\item A path's codomain is its endpoint: $\cod([f]) = y$.
	\item The identity is the constant map: $1_x = [c_x]$.
	\item Composition is concatenation: $[g][f] = [f*g]$.
\end{itemize}

This construction is well-defined specifically because we are working with path
homotopies. For example, in general two paths with different start points may be
free homotopic, meaning without restricting to path homotopy we could not even
write down the domain and codomain of our morphisms.

\begin{ex}\cite[p. 213]{Brown}
	We can immediately compute a few fundamental groupoids.
	\begin{itemize}
		\item The fundamental groupoid of a convex space is a tree groupoid (defined
		      in \Cref{groupoids}). This corresponds to the fact that any two paths with
		      the same endpoints in such a space are homotopic via the straight line
		      homotopy.
		\item The fundamental groupoid of a totally disconnected space is a discrete
		      groupoid. This corresponds to the fact that the only paths in such spaces
		      are the constant paths.
	\end{itemize}
\end{ex}

\subsection{Categorical Properties} We can learn a lot about the fundamental
groupoid by studying it categorically. First, we should confirm it is what we
claim it is:

\begin{prop}\label{fundamental groupoid is a groupoid}
	The fundamental groupoid is a groupoid.
\end{prop}

\begin{proof}
	All of this work was already done in class for the fundamental group. We
	restate the results here for groupoids.
	\begin{itemize}
		\item Composition is well-defined, since concatenation preserves homotopy
		      class.
		\item Composition is associative, since concatenation is associative up to
		      homotopy.
		\item Every object $x$ has $[c_x]$ as an identity.
		\item Every morphism $[f]$ has $[\bar{f}]$ as an inverse.
	\end{itemize}

	The first three say that $\Pi_1X$ is a category, and the last says that it
	is a groupoid.
\end{proof}

The construction of the fundamental groupoid naturally gives rise to a functor
$$\Pi_1: \cat{Top}\rightarrow\cat{Grpd}.$$ In particular, let $f: X\rightarrow
	Y$ be a continuous function. We can view $f$ as acting on paths via composition.
Accordingly, we define
\begin{align*}
	\Pi_1f  \colon \Pi_1X & \to \Pi_1Y              \\
	[\gamma]              & \mapsto [f\circ\gamma].
\end{align*}

% TODO: picture

This mapping is well-defined because composition preserves homotopy equivalence.

% TODO: output is a homomorphism

\begin{prop}\label{fundamental groupoid is a functor}
	$\Pi_1$ is a functor.
\end{prop}

\begin{proof}
	Again, much of this work was done in class.
	\begin{itemize}
		\item $\Pi_1$ respects composition, since composition is associative.
		\item $\Pi_1$ respects identities, since composing the identity fixes homotopy classes. \qedhere
	\end{itemize}
\end{proof}

This result is an improvement over the fundamental group, where we needed a
functor out of based spaces for the definition to make sense. This is a first
hint that the fundamental groupoid in some sense captures more of the structure
of a space than the fundamental group does.

\begin{cor}\label{fundamental groupoid is topological}
	The fundamental groupoid is a topological invariant. More precisely, if
	$X\cong Y$, then $\Pi_1X\cong \Pi_1Y$.
\end{cor}

\begin{proof}
	This follows from \Cref{functors preserve isomorphism} and \Cref{fundamental
		groupoid is a functor}.
\end{proof}

\begin{rem}\label{fundamental groupoid is not homotopy invariant} As defined,
	the fundamental groupoid is a \emph{not} a homotopy invariant. For instance,
	since objects of the fundamental groupoid are in bijection with elements of
	the underlying space, the fundamental groupoid of $D^n$ has uncountably many
	objects, whereas the fundamental groupoid of the point has only one. \end{rem}

This unfortunate fact suggests an issue with our notion of isomorphism of
categories: it requires objects to be in bijection, which is far too strong to
express homotopy invariants. There is a more natural notion, equivalence of
categories, somewhat analogous to homotopy equivalence, which requires some
additional machinery to develop.

\section{Natural Transformations and 2-Categories}
\label{natural transformations}

If we want our categorical constructions to play nicely with homotopy, it will
help to define an analogue to homotopy on categories. It turns out the correct
notion is a \emph{natural transformation}, a way to relate functors analogous to
a homotopy. This notion will allow us to define \emph{equivalence of
	categories}. Here I follow \cite[Sections 1.4-1.7]{Riehl}.

\subsection{Natural Transformations as Categorical Homotopy}

Let $H$ be a homotopy between continuous maps $f,g: X\rightarrow Y$. We can think
of $H(x, -)$ as morphing the point $f(x)$ into $g(x)$. In the same way, we want to define
a transformation between functors $F,G: \cat{C}\rightarrow\cat{D}$ which we can
think of as morphing the object $Fx$ into the point $Gx$. Of course, the right
thing to do this morphing is exactly a morphism in $\cat{D}$. That motivates the
following definition:

\begin{defn}[Natural Transformation]\label{natural transformation}
	A \emph{natural transformation} $\alpha: F\nat G$ between
	functors $F,G:\cat{C}\rightarrow\cat{D}$ consists of, for each
	object $x\in\Obj{C}$, a morphism $\alpha_x: F_x\rightarrow G_x\in \cat{D}$,
	called the \emph{components} of $\alpha$.

	This data must preserve morphisms in $\cat{C}$, in the sense that for any
	morphism $f: x\rightarrow y\in\cat{C}$, the following square must commute:
	\begin{figure}[H]
		\centering
		\begin{tikzcd}
			Fx & Gx \\
			Fy & Gy
			\arrow["Ff", from=1-1, to=2-1, swap]
			\arrow["\alpha_x", from=1-1, to=1-2]
			\arrow["\alpha_y", from=2-1, to=2-2, swap]
			\arrow["Gf"', from=1-2, to=2-2, swap]
		\end{tikzcd}
	\end{figure}
\end{defn}

Thinking of natural transformations as categorical analogues of homotopy will be
critical for what is to come. The commutative square above is analogous to the
requirement that a homotopy behaves as $f$ at $t=0$ and as $g$ at $t=1$; we
require that the transformation behaves as $F$ "before" the transformation and
as $G$ "after" the transformation.

An alternative definition makes the translation even clearer. Let $\mathbbm{2}$
be the category with two objects, $0$ and $1$, and a single non-identity
morphism $0\rightarrow 1$. Defining the product category in the obvious way,
with component-wise composition, a natural transformation between $F$ and $G$
corresponds bijectively to a functor $H:
	\cat{C}\times\mathbbm{2}\rightarrow\cat{D}$ such that the following diagram
commutes:

\begin{figure}[H]
	\centering
	\begin{tikzcd}
		\cat{C} & \cat{C}\times\mathbbm{2} & \cat{C} \\
		& \cat{D} &
		\arrow["i_0", from=1-1, to=1-2]
		\arrow["i_1", from=1-3, to=1-2, swap]
		\arrow["F", from=1-1, to=2-2, swap]
		\arrow["G", from=1-3, to=2-2]
		\arrow["H", from=1-2, to=2-2]
	\end{tikzcd}
\end{figure}

Here $i_0$ and $i_1$ are the obvious inclusion functors. (TODO: proof from
\cite[Lemma 1.5.1]{Riehl}.)

One key difference between homotopy and natural transformations is that
homotopies are always invertible, in the sense that for a homotopy $H$ from $f$
to $g$, $H(-, 1-t)$ is a homotopy from $g$ to $h$. This key fact means
homotopies can be used to define an equivalence relation on continuous maps.
Since we are primarily concerned with the special case of groupoids, in which
"everything" is invertible, this ought to be a non-issue.

\begin{defn}
	A \emph{natural isomorphism} is a natural transformation whose components are
	each isomorphisms.
\end{defn}

\begin{rem}
	Any natural transformation into a groupoid is a natural isomorphism.
\end{rem}

\begin{thm}\label{natural isomorphisms are invertible}
	Let $F,G: \cat{C}\rightarrow\cat{D}$ be functors, and $\alpha$ a natural
	isomorphism between them. For each $x\in\cat{D}$, let $\beta_x$ be the
	inverse of $\alpha_x$. Then the $\beta_x$ collect into a natural
	isomorphism, $\beta$, between $\cat{D}$ and $\cat{C}$.
\end{thm}

\begin{proof}
	That $\beta$ is an isomorphism follows because inverses of isomorphisms are
	isomorphisms. We need to show that $\beta$ is a natural transformation, i.e.
	that for any $x, y\in\cat{C}$, the following square commutes:

	\begin{figure}[H]
		\centering
		\begin{tikzcd}
			Gx & Fx \\
			Gy & Fy
			\arrow["Gf", from=1-1, to=2-1, swap]
			\arrow["\beta_x", from=1-1, to=1-2]
			\arrow["\beta_y", from=2-1, to=2-2, swap]
			\arrow["Ff", from=1-2, to=2-2]
		\end{tikzcd}
	\end{figure}

	But this exactly comes from replacing $\alpha_x$ and $\alpha_y$ with their
	inverses $\beta_x$ and $\beta_y$ in the square from \Cref{natural
		transformation}, which we can do by definition of isomorphism.

\end{proof}

\subsection{Equivalence of Categories}

TODO.

\subsection{The 2-Category of Categories.}\label{1cat}

TODO.

\section{Higher Categories and Higher Groupoids}\label{higher categories}

TODO.

\section{$\infty$-groupoids as spaces}\label{infinity groupoids}

TODO; not sure I'll have space for this. Following \cite{Porter}.

\bibliographystyle{alpha}
\bibliography{references}

\end{document}
