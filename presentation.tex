\documentclass{beamer}
\usepackage{adjustbox}
\usepackage{multicol}
\usepackage{tikz-cd}
\usepackage{quiver}
\usepackage{bbm}

\theoremstyle{definition}
\newtheorem{defn}{Definition}

\setbeamertemplate{caption}{\raggedright\insertcaption\par}

%% ALPHABETS %%
\def\cA{\mathcal{A}}
\def\cB{\mathcal{B}}
\def\cC{\mathcal{C}}
\def\cD{\mathcal{D}}
\def\cE{\mathcal{E}}
\def\cF{\mathcal{F}}
\def\cG{\mathcal{G}}
\def\cH{\mathcal{H}}
\def\cI{\mathcal{I}}
\def\cJ{\mathcal{J}}
\def\cK{\mathcal{K}}
\def\cL{\mathcal{L}}
\def\cM{\mathcal{M}}
\def\cN{\mathcal{N}}
\def\cO{\mathcal{O}}
\def\cP{\mathcal{P}}
\def\cQ{\mathcal{Q}}
\def\cR{\mathcal{R}}
\def\cS{\mathcal{S}}
\def\cT{\mathcal{T}}
\def\cU{\mathcal{U}}
\def\cV{\mathcal{V}}
\def\cW{\mathcal{W}}
\def\cX{\mathcal{X}}
\def\cY{\mathcal{Y}}
\def\cZ{\mathcal{Z}}

\def\AA{\mathbb{A}}
\def\BB{\mathbb{B}}
\def\CC{\mathbb{C}}
\def\DD{\mathbb{D}}
\def\EE{\mathbb{E}}
\def\FF{\mathbb{F}}
\def\GG{\mathbb{G}}
\def\HH{\mathbb{H}}
\def\II{\mathbb{I}}
\def\JJ{\mathbb{J}}
\def\KK{\mathbb{K}}
\def\LL{\mathbb{L}}
\def\MM{\mathbb{M}}
\def\NN{\mathbb{N}}
\def\OO{\mathbb{O}}
\def\PP{\mathbb{P}}
\def\QQ{\mathbb{Q}}
\def\RR{\mathbb{R}}
\def\SS{\mathbb{S}}
\def\TT{\mathbb{T}}
\def\UU{\mathbb{U}}
\def\VV{\mathbb{V}}
\def\WW{\mathbb{W}}
\def\XX{\mathbb{X}}
\def\YY{\mathbb{Y}}
\def\ZZ{\mathbb{Z}}

%% OTHER SYMBOLS %% 
\DeclareMathOperator{\id}{id}
\DeclareMathOperator{\dom}{dom}
\DeclareMathOperator{\cod}{cod}

\def\nat{\Rightarrow}
\def\monic{\rightarrowtail}
\def\epic{\twoheadrightarrow}
\def\pathto{\rightsquigarrow}

\newcommand{\cat}[1]{{\normalfont\texttt{#1}}}
\newcommand{\op}[1]{{{#1}^{\cat{op}}}}
\newcommand{\opc}[1]{\op{\cat{#1}}}
\newcommand{\Obj}[1]{\text{Obj}(\cat{#1})}
\newcommand{\Map}[1]{\text{Map}(\cat{#1})}

% https://tex.stackexchange.com/a/118217
% \DeclarePairedDelimiter\ceil{\lceil}{\rceil}
% \DeclarePairedDelimiter\floor{\lfloor}{\rfloor}

\title{The Fundamental Groupoid}
\author{Riley Shahar}
\begin{document}

\tikzset{
	invisible/.style={opacity=0},
	visible on/.style={alt={#1{}{invisible}}},
	alt/.code args={<#1>#2#3}{%
			\alt<#1>{\pgfkeysalso{#2}}{\pgfkeysalso{#3}}%
		}
}

\date{}
\frame{\titlepage}

\begin{frame}[fragile]
	\frametitle{Categories}

	\begin{multicols}{2}
		\begin{tikzcd}[node distance=2cm,auto]
			\only<2->{x}
			\arrow[visible on=<3-3>, rd, swap, "h"]
			\arrow[visible on=<4->, rd ,swap, "gf = h"]
			\arrow[visible on=<3->, r, "f"]
			\arrow[visible on=<5->, out=155, in=115, loop, "1_x"]
			&
			\only<2->{y}
			\arrow[visible on=<3->, d, "g"]
			\arrow[visible on=<5->, out=65, in=25, loop, "1_y"] \\
			&
			\only<2->{z}
			\arrow[visible on=<5->, loop right,"1_z"]
		\end{tikzcd}

		\begin{itemize}
			\onslide<2->{\item Objects}
			      \onslide<3->{\item Morphisms}
			      \onslide<4->{\item Composition}
			      \onslide<5->{\item Identities}
		\end{itemize}
	\end{multicols}
\end{frame}

\begin{frame}[fragile]
	\frametitle{Category Axioms}
	\begin{multicols}{2}
		\begin{figure}
			\begin{tikzcd}
				\only<2->{x}
				\arrow[visible on=<2->, "f", r]
				\arrow[visible on=<2->, "gf"', bend right=30, rr]
				&
				\only<2->{y}
				\arrow[visible on=<2->, "g", r]
				\arrow[visible on=<2->, "hg", bend right=-30, rr]
				&
				\only<2->{z}
				\arrow[visible on=<2->, "h", r]
				&
				\only<2->{w}
			\end{tikzcd}
			\only<2->{\caption{Associative}}
		\end{figure}
		\begin{figure}
			\begin{tikzcd}
				\only<3->{x}
				\arrow[visible on=<3->, "1_x", r]
				\arrow[visible on=<3->, "f"', dr]
				&
				\only<3->{x}
				\arrow[visible on=<3->, "f", d]
				\\
				&
				\only<3->{y}
			\end{tikzcd}
			\begin{tikzcd}
				\only<3->{x}
				\arrow[visible on=<3->, "f", r]
				\arrow[visible on=<3->, "f"', dr]
				&
				\only<3->{y}
				\arrow[visible on=<3->, "1_y", d]
				\\
				&
				\only<3->{y}
			\end{tikzcd}
			\only<3->{\caption{Unital}}
		\end{figure}
	\end{multicols}
\end{frame}

\begin{frame}[fragile]
	\frametitle{Examples}
	\pause
	\begin{itemize}
		\item \cat{Set}: Sets and Functions \pause
		\item \cat{Grp}: Groups and Homomorphisms \pause
		\item \cat{Top}: Spaces and Continuous Functions
	\end{itemize}
\end{frame}

\begin{frame}[fragile]
	\frametitle{Groups as Categories}

	\pause

	Take a group $G$. Make a category $\GG$: \pause
	\begin{itemize}
		\item Objects\only<4->{: There is a single object, $*$.}
		\item Morphisms\only<5->{: For each $x\in G$, there is a morphism
			      $\mathbbm{x}$.}
		\item Composition\only<6->{: Composition is the group multiplication.}
		\item Identities\only<7->{: The identity $1_*$ is the morphism
			      $\mathbbm{e}$.}
	\end{itemize}

	\uncover<8->{Moral: not all categories have structured sets for objects and
		structure-preserving functions for morphisms.}
\end{frame}

\begin{frame}[fragile]
	\frametitle{Isomorphisms}

	\pause

	Goal: generalize bijections, group isomorphisms, homeomorphisms etc. \pause

	\begin{defn}
		A morphism $f: x\rightarrow y$ is an \emph{isomorphism} when there exists an
		\emph{inverse morphism} $g: y\rightarrow x$ such that
		$$fg=1_y\quad\text{and}\quad gf = 1_x.$$
	\end{defn}
	\pause

	What are isomorphisms in $\GG$? \pause Everything!
\end{frame}

\begin{frame}[fragile]
	\frametitle{Groupoids}

	What are isomorphisms in $\GG$? Everything! \pause

	Let's generalize! \pause

	\begin{defn}
		A \emph{groupoid} is a category in which every morphism is an isomorphism.
	\end{defn}

	\pause
\end{frame}


\usetikzlibrary{decorations.markings}

\newcommand{\pathA}{(2, 2) .. controls (3,2) and (3, 1.5) .. (4, 1.5)}
\newcommand{\pathB}{(4, 1.5) .. controls (5, 1.5) and (5,2) .. (6, 2)}
\newcommand{\pathC}{(6, -2) .. controls (5,-2) and (5, -1.5) .. (4, -1.5)}
\newcommand{\pathD}{(4, -1.5) .. controls (3, -1.5) and (3,-2) .. (2, -2)}

\begin{frame}[fragile]
	\frametitle{Topology??}
	\begin{adjustbox}{max totalsize={.9\textwidth}{.7\textheight},center}
		\begin{tikzpicture}
			% outer shape
			\filldraw[fill=black, fill opacity=0.2, draw=black, visible on=<2-3>] (0, 0) arc(180:90:2) --
			\pathA --
			\pathB --
			(6, 2) arc(90:0:2) --
			(8, 0) arc(360:270:2) --
			\pathC --
			\pathD --
			(2, -2) arc(270:180:2)
			;

			% inner holes
			\filldraw [visible on=<2-3>, fill=white] (2,0) circle (1);
			\filldraw [visible on=<2-3>, fill=white] (6,0) circle (1);

			% paths
			\coordinate (x) at (3.4, 0.6);
			\coordinate (y) at (4.6, -0.6);
			\node [visible on=<3->, circle, fill, inner sep=2pt] at (x) {};
			\draw[visible on=<3->, <->] ([shift=(37:1.5)]2,0) arc(37:370:1.5);
			\draw[visible on=<3->, <-] (3.65, 0.8) .. controls (4,1) and (5,1.5) .. (6, 1.5);
			\draw[visible on=<3->, <-] (4.8, -0.83) .. controls (5.2,-1.2) and (5.3, -1.5) .. (6, -1.5);
			\draw[visible on=<3->] ([shift=(270:1.5)]6,0) arc(270:450:1.5);

			\node [visible on=<3->, circle, fill, inner sep=2pt] at (y) {};
			\draw[visible on=<3->, <->] (3.6, 0.4) -- (4.4, -0.4);

		\end{tikzpicture}
	\end{adjustbox}

	\uncover<5->{Some dots and arrows between them?? That sounds like..}

	\uncover<6->{a category!}

	\uncover<7->{If you relax the requirement that paths in the fundamental group
		are loops, you get \emph{the fundamental groupoid}.}
\end{frame}
\end{document}
